\documentclass{article}
\usepackage[T1,T2A]{fontenc}
\usepackage[utf8]{inputenc}
\usepackage[english]{babel}
\usepackage{amsmath}
\usepackage{siunitx}
\begin{document}
\title{Flight Model}
\author{Vasili V}
\renewcommand{\today}{November 22, 2018}
\maketitle
\section{Lift}
The buoyant force on a balloon of diameter $d$ with density of content $\rho_{in}$ and environmental density $\rho_{out}$:
\begin{equation}
\begin{aligned}
F_b &= (\rho_{out} - \rho_{in})g\frac{\pi{d}^3}{6},
\end{aligned}
\end{equation}
where content is Helium with density (at STP):
\begin{equation}
\begin{aligned}
\rho_{in} &= 0.1786 && [\si{kg.m^{-3}}].
\end{aligned}
\end{equation}
The environment is standard atmosphere taken from \cite{NASA76} at zero altitude:
\begin{equation}
\begin{aligned}
\rho_{out} &= 1.225 && [\si{kg.m^{-3}}].
\end{aligned}
\end{equation}
under following conditions:
\begin{equation}
\begin{aligned}
p &= \num{1.01325e5} && [\si{kg.m^{-3}}]. \\
T &= \num{288.15} && [\si{\kelvin}].
\end{aligned}
\end{equation}
Under the conditions Helium density is:
\begin{equation}
\begin{aligned}
\rho_{in} &= 0.1786\cdot\frac{\num{1.01325e5}}{\num{1e5}}\cdot\frac{\num{273.15}}{\num{288.15}} = 0.1715 && [\si{kg.m^{-3}}].
\end{aligned}
\end{equation}
Let's simplify equation (1):
\begin{equation}
\begin{aligned}
F_b &= k_0d^3,
\end{aligned}
\end{equation}
where $k_0$ is:
\begin{equation}
\begin{aligned}
k_0 &= \frac{\pi}{6}\cdot\num{9.81}\cdot(\num{1.225}-\num{0.1715}) = 5.411. && [\si{\newton.m^{-3}}]
\end{aligned}
\end{equation}
For example balloons with diameters $\num{0.36}$, $\num{0.70}$ and $\num{1.15}$ $\si{m}$ get following lifts:
\begin{equation}
\begin{aligned}
F_s &= \num{5.411}\cdot\num{0.36}^3 = 0.2525 [\si{\newton}] = 25.74 [\si{gf}]; \\
F_m &= \num{5.411}\cdot\num{0.70}^3 = 1.856 [\si{\newton}] = 189.3 [\si{gf}]; \\
F_b &= \num{5.411}\cdot\num{1.15}^3 = 8.230 [\si{\newton}] = 839.2 [\si{gf}].
\end{aligned}
\end{equation}
These balloons have following volumes:
\begin{equation}
\begin{aligned}
V_s &= 24.43 && [\si{L}]; \\
V_m &= 179.6 && [\si{L}]; \\
V_b &= 796.3 && [\si{L}],
\end{aligned}
\end{equation}
which is approximately $\frac{1}{57}$, $\frac{1}{7}$ and $\frac{2}{3}$ of 10 L tank.
\begin{thebibliography}{9}
\bibitem{NASA76}
  NOAA, NASA, USAF,
  US. Standard Atmosphere,
  1976
\end{thebibliography}
\end{document}
